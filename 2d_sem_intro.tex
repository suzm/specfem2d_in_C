  \documentclass{book}
  \usepackage{amsmath}
  \titlepage
  \title{2D Spectral Element Method}
  \begin{document}
  %\chapter{}
  %\section{General Introduciton}
  The primary goal of this introduction is to demonstrate how to implement spectral element method in 2d. We seek to produce the displacement field caused by an earthquake in a finite earth model with volume $\Omega$. In order to simplify calculation, we just use stress-free boundary condition.  We will  talk about artificial absorbing  boundary condition in detail later.
  The displacement field $\boldsymbol{u}$ is produced by an earthquake is governed by the momentum equation,
  \begin{equation}
  \rho\,\partial_t^2\boldsymbol{u}=\boldsymbol{\nabla}\cdot\boldsymbol{T}+\boldsymbol{f}
  \end{equation}
  ,where $\boldsymbol T =(\boldsymbol T_1,\boldsymbol T_2,\boldsymbol T_3)^T$, and $\boldsymbol T_i$ is vector function.
  The weak form of above equation is produced by dotting the momentum equation with an arbitrary test vector $\boldsymbol{w}=(w_1(\boldsymbol x),w_2(\boldsymbol x),w_3(\boldsymbol x))^T $, integrating by the model volume $\Omega$. Since we use stress-free boundary condition, which is easy to derive the weak form of wave equation, we have $\boldsymbol T|_{\partial \Omega} = 0$
  \begin{equation}
  \begin{aligned}
  &\int_\Omega \boldsymbol{\nabla}\cdot\boldsymbol{T}\cdot\boldsymbol w(\boldsymbol{x})\;\mathrm{d}^3\boldsymbol{x}\\
  & = \sum _{i=1}^3\int_\Omega w_i(\boldsymbol{x})\nabla\cdot\boldsymbol{T}_i\;\mathrm{d}^3\boldsymbol{x}\\
  & = \sum_{i=1}^3\int_\Omega \nabla\cdot(w_i(\boldsymbol{x})\boldsymbol{T}_i) - \nabla w_i(\boldsymbol x) \cdot\boldsymbol T_i\;\mathrm{d}^3\boldsymbol{x}\\
  & = \sum_{i=1}^3\left(\int_{\partial\Omega} w_i(\boldsymbol{x})\boldsymbol T_i \cdot \;\mathrm{d}\boldsymbol{s}-\int_\Omega\nabla w_i(\boldsymbol x) \cdot\boldsymbol T_i\;\mathrm{d}^3\boldsymbol{x} \right)\\
  & = -\sum_{i=1}^3\int_\Omega\nabla w_i(\boldsymbol x) \cdot\boldsymbol T_i\;\mathrm{d}^3\boldsymbol{x}\\
  & = -\int_\Omega \boldsymbol{\nabla}\boldsymbol w(\boldsymbol x): \boldsymbol T\;\mathrm{d}^3\boldsymbol x
  \end{aligned}
  \end{equation}

The Source function is 
\begin{equation}
 \boldsymbol{f} = -\boldsymbol{M}\cdot\boldsymbol\nabla\delta(\boldsymbol x-\boldsymbol x_s)S(t)
\end{equation}
Using the properties of Dirac delta distribution,after integration, it transformed in the following way,
\begin{equation}
\begin{aligned}
 &\int_\Omega \boldsymbol f\cdot \boldsymbol w\;\mathrm{d}^3\boldsymbol x\\
 &= -S(t)\int_\Omega\boldsymbol{M}\cdot \boldsymbol\nabla\delta(\boldsymbol x-\boldsymbol x_s)\cdot \boldsymbol w\;\mathrm{d}^3\boldsymbol x\\
 &= \boldsymbol{M}:\boldsymbol{\nabla w}(\boldsymbol x_s)S (t)
\end{aligned}
\end{equation}

In 2 dimension, We can derive this using Green Formula. 
\begin{equation}
 \int_\Omega \rho\,\boldsymbol w\cdot \partial_t^2\boldsymbol u\;\mathrm{d}^2\boldsymbol x = -\int_\Omega \boldsymbol{\nabla}\boldsymbol w(\boldsymbol x): \boldsymbol T\;\mathrm{d}^2\boldsymbol x + \boldsymbol{M}:\boldsymbol{\nabla w}(\boldsymbol x_s)S (t)
\end{equation}
In the following parts, we will explain the general procedure of specfem.

First, we divide the region into  a number of non-overlapping elements, $\Omega_e$, $e=1, \ldots, n_e$, such that $\Omega=\bigcup_e^{n_e}\Omega_e$, as shown in Fig.1. In SEM, the shape of element is restricted to quadrilateral. The integration in the whole region become the sum of integration over every element, as shown below.
\begin{equation}
 \sum_e^{n_e}\int_{\Omega_e}  \rho\,\boldsymbol w\cdot \partial_t^2\boldsymbol u\;\mathrm{d}^2\boldsymbol x = -\sum_e^{n_e}\int_{\Omega_e} \boldsymbol{\nabla}\boldsymbol w(\boldsymbol x): \boldsymbol T\;\mathrm{d}^2\boldsymbol x + \boldsymbol{M}:\boldsymbol{\nabla w}(\boldsymbol x_s)S (t)
\end{equation}

Since the equation holds for any test function $\boldsymbol w$, we can use $\boldsymbol w$ to extract the value of displacement$\boldsymbol u$ at one point by setting $w$ equals 1 at this point and equals 0 at other points. It's undoubted that this kind of $\boldsymbol w$ did exist. Using Gauss-Lobatto-Legendre points, the integration become the forms like below:
\begin{equation}
 M\ddot U+K = F,
\end{equation}
where $M$ denotes the global mass matrix, $K$ the global stiffness matrix and $F$ the source term.
Let's build the mass matrix $M$ first. According the quadrature, we can know that the mass matrix is diagonal. 











  \end{document}
